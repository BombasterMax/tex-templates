\documentclass[a4paper]{article}
\usepackage[14pt]{extsizes}
\usepackage[utf8x]{inputenc}
\usepackage[russian]{babel}
\usepackage{setspace}
\onehalfspacing
\usepackage{indentfirst}
\usepackage{fontspec}
\setlength{\parindent}{12.5mm}
\usepackage{multirow}
\usepackage{tabularx}
\usepackage{pdflscape}
\usepackage{geometry}
\geometry{top=2cm}
\geometry{right=1cm}
\geometry{left=3cm}
\geometry{bottom=2cm}
\setmainfont{Liberation Serif}
\setsansfont{DejaVu Sans}
\usepackage{amsmath}
\usepackage{amsfonts}
\usepackage{graphicx}
\usepackage{xcolor}
\usepackage{xlop}
\usepackage{amsmath}
\date{}

\begin{document}

\begin{titlepage}
    \newpage
    
    \begin{center}
    
    {\bfseries МИНОБРНАУКИ РОССИИ \\
    САНКТ-ПЕТЕРБУРГСКИЙ ГОСУДАРСТВЕННЫЙ\\
    ЭЛЕКТРОТЕХНИЧЕСКИЙ УНИВЕРСИТЕТ\\
    «ЛЭТИ» ИМ. В.И. УЛЬЯНОВА (ЛЕНИНА)}\\
    

    \vspace{0.25cm}
    \end{center}
    \begin{flushright}
    	\begin{center}
    		\textbf{\textcolor{red}{Название кафедры}}
    	\end{center}
    \end{flushright}  
    

    \vspace{5cm}

    \begin{center}

    \bfseries ОТЧЕТ \linebreak
    по лабораторной работе № \linebreak
    по дисциплине <<\textcolor{red}{название дисциплины}>> \linebreak
    Тема: \textcolor{red}{Название темы}
    \end{center}

    \vspace{3cm}

   Студент \textcolor{red}{(ка)} гр. \textcolor{red}{xxxx} \hspace{7.07cm} \textcolor{red}{Фамилия И.О.}

   Преподаватель ~~~\hspace{8cm} \textcolor{red}{Фамилия И.О.}

    \vfill
    
    \centering Санкт-Петербург \\ 2023

\end{titlepage}
\newpage
\setcounter{page}{2}
\section*{\large Основные теоретические положения}
\section*{\large Ход работы}
\section*{\large Выводы}
\end{document}
